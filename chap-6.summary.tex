\section{Summary of results}
This chapter shows how results of software testing can be used to estimate software reliability.
The main observations of the study are:
\begin{enumerate}
\item The test adequacy is the major factor in determining the software reliability in systems
related to safety.
\item The estimated software reliability is a function of test adequacy $(adequacy(T)$),
the amount of verification carried out $(x)$, and the amount of verified code reused $(y)$.
%\item The test adequacy is the limiting factor for achieving the software
%reliability in systems important to safety. \todo{Fix}
\item For a given software reliability target, as the value of $x$ increases, the requirement
for $y$ decreases. The decrease in required value of $y$ exhibits
linear to exponential behavior as the target reliability increases,
and becomes a step function as the $Reliability \rightarrow adequacy(T)$ and
$x\rightarrow1$.
\item The proposed approach re-iterates the fact that: achieving high reliability
for software with high reusability is relatively easier.
\item For software with high test adequacy, values of $x$,
$y$ may give some insights on properties of the software.
\item The probability of software failure in the case studies have been found to be lesser than 10$^{-5}$ for a random input from the input domain.
\end{enumerate}

%The approaches presented in the chapter can be used by safety-critical software developers to improve the software reliability. Also, regulators may use the techniques to verify reliability, safety and dependability claims. \todo{REPEATED}
